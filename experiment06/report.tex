\documentclass[a4paper,12pt]{article}
\usepackage[super,square]{natbib}
% Package to change margin size
\usepackage{anysize}
\usepackage{amsmath}
\marginsize{2cm}{2cm}{1cm}{2cm}
% Package to make headers
\usepackage{fancyhdr}
\usepackage{circuitikz}
\renewcommand{\headrulewidth}{0pt}
\usepackage{soul}
\usepackage[section]{placeins}
% Colors for the references links
\usepackage[dvipsnames]{xcolor}
% Package to link references
\usepackage{hyperref}
\usepackage{graphicx}
\usepackage{float}
\hypersetup{
 colorlinks=true,
  linkcolor=black,
  citecolor=CadetBlue,  
  filecolor=CadetBlue,      
  urlcolor=CadetBlue,
}
% Package for lorem ipsum
\usepackage{lipsum}
% Package for multicolumn
\usepackage{multicol}
% Package for removing paragraph identations
\usepackage{parskip}
\setlength\columnsep{18pt}
% Sets bastract
\renewenvironment{abstract}
 {\par\noindent\textbf{\abstractname}\ \ignorespaces \\}
 {\par\noindent\medskip}



 
\begin{document}
% Makes header
\pagestyle{fancy}
\thispagestyle{empty}
\fancyhead[R]{\textit{EE1200}}
\fancyhead[L]{}
% Makes footnotes with an asterisk
\renewcommand*{\thefootnote}{\fnsymbol{footnote}}
\begin{center}
\Large{\textbf{Experiment 06}}
\vspace{0.4cm}
\normalsize
\\ Aditya Tripathy - ee24btech11001, Durgi Swaraj Sharma - ee24btech11018\\
\medskip
\normalsize
\end{center}
{\color{gray}\hrule}
\vspace{0.4cm}
\begin{abstract}
In Experiment-06, we used a Op-amps to make Sallen-Key low pass filter, a Sallen-key high pass filter, and a band pass filter by cascading them. We then compared and verified them with the theoretical results. 
\end{abstract}
{\color{gray}\hrule}
\medskip
\section{Objective}
\begin{itemize}
	\item To design and implement a bandpass filter using separate Sallen-Key Low Pass Filter (LPF) and High Pass Filter (HPF).
	\item To analyze and compare the frequency response of LPF, HPF, and the final bandpass filter.
	\item To plot the magnitude response (gain vs. frequency) of all three filter.
\end{itemize}
\section{Apparatus}
\begin{itemize}
	\item Op-amp (we used the IC LM358 as it has 2 Op-amps built-in)
	\item Resistors [$R = 22k\Omega, R_1 = 5.6k\Omega$]
	\item Capacitors (2 per filter, 4 for cascade) [$C = 220nF$]
	\item Oscilloscope
	\item Function generator
	\item DC power supply (to power the IC)
	\item Connecting wires and probes, breadboard
\end{itemize}
\section{Procedure}
\begin{enumerate}
	\item IC
		\begin{itemize}
			\item Use the datasheet of your Op-amp to know the function of its pins and specifications.
			\item Ensure the supply voltages $V_+$ and $V_-$ being provided are within the recommended range in the datasheet, before you turn them on (We used $+15V$ and $-15V$).
		\end{itemize}
	\item Prepating the Circuits (same procedure for all three circuits)
		\begin{itemize}
			\item Make the circuits as shown in the theory section using the right pins of the Op-amp, resistors, and connecting wires. 
			\item Connect $V_{out}$ to ground through a resistor. 
			\item Apply the sine wave to $V_{in}$ from the function generator. 
		\end{itemize}
	\item Oscilloscope: Use the probes to measure input and output voltages, $V_{in}$ and $V_{out}$. Set up the Measure function to measure the $V_{pp}$s of the two signals.
	\item Taking Measurements
		\begin{itemize}
			\item Start with a small frequency of the input sine wave. Note down the $V_{pp}$ values on the oscilloscope. 
			\item Gradually increase the frequency of the input sine wave.
			\item In intervals, note down the $V_{pp}$s being measured on the oscilloscope and the frequencies they occur at. The ratio of the $V_{pp}$s is the voltage gain, and can be used to plot this data when verifying.
			\item As the frequency increases, you might need to make bigger jumps to see meaningful changes in the data.
			\item In high-pass filters, at higher frequencies the $V_{out}$ saturates to the value of $V_{in}$, so one may stop measuring after that.
			\item Continue measuring until enough data is obtained for verification.
		\end{itemize}
	\item Cascading to obtain a Band Pass Filter.
		\begin{itemize}
			\item Using the prepared low pass and high pass filter, we design a band pass filter by cascading them.
			\item Connect the $V_{out}$ of the low pass filter to the $V_{in}$ of the high pass filter. Remove the previous input signal from the $V_{int}$ of the highpass filter. 
		\end{itemize}
	\item Plotting: Use a plotting software to plot the measured data with the theoretical response for the sake of verification.
\end{enumerate}
\end{document}
